\documentclass{article}

\usepackage[a4paper, total={7in, 9in}]{geometry}
\pagestyle{empty}

\usepackage{xcolor} % Required for listings color definitions
\definecolor{Brown}{cmyk}{0,0.81,1,0.60}
\definecolor{OliveGreen}{cmyk}{0.64,0,0.95,0.40}
\definecolor{CadetBlue}{cmyk}{0.62,0.57,0.23,0}
\definecolor{lightlightgray}{gray}{0.9}

\usepackage{listings} % computer code language formatting

\lstdefinestyle{tex-style} {
	language=[LaTeX]TeX,                    % Code langugage
	basicstyle=\ttfamily,                   % Code font, Examples: \footnotesize, \ttfamily
	%keywordstyle=\color{OliveGreen},        % Keywords font ('*' = uppercase)
	commentstyle=\color{gray},              % Comments font
	numbers=none,                           % Line nums position
	numberstyle=\tiny,                      % Line-numbers fonts
	stepnumber=1,                           % Step between two line-numbers
	numbersep=5pt,                          % How far are line-numbers from code
	backgroundcolor=\color{lightlightgray}, % Choose background color
	frame=single,                             % A frame around the code
	tabsize=2,                              % Default tab size
	captionpos=b,                           % Caption-position = bottom
	breaklines=true,                        % Automatic line breaking?
	breakatwhitespace=false,                % Automatic breaks only at whitespace?
	showspaces=false,                       % Dont make spaces visible
	showtabs=false,                         % Dont make tabls visible
	columns=flexible,                       % Column format
	morekeywords={__global__, __device__}  % CUDA specific keywords
}
\lstnewenvironment{latex}
{\lstset{language=[LaTeX]TeX}}
{}
\lstset{style=tex-style}

\usepackage[hyphens]{url}
\usepackage{hyperref}
\hypersetup{
	colorlinks=true,
	citecolor=black,
	filecolor=black,
	linkcolor=blue,
	urlcolor=blue
}

\begin{document}
	
	\section*{Template}
	
	\subsection*{Description}
	A uniform template is useful to keep all the tips and examples of this portfolio consistent and searchable. The related code, both for the preamble and the document, as well as the compiled examples are provided in separate sections.
	
	For this section, the template code is presented.
		
	\subsection*{Used Packages}
	\verb|xcolor, listings, url, hyperref|
	
	\subsection*{Preamble}
	\begin{lstlisting}
		\documentclass{article}
		
		\usepackage[a4paper, total={7in, 9in}]{geometry}
		\pagestyle{empty}
		
		\usepackage{xcolor} % Required for listings color definitions
		\definecolor{Brown}{cmyk}{0,0.81,1,0.60}
		\definecolor{OliveGreen}{cmyk}{0.64,0,0.95,0.40}
		\definecolor{CadetBlue}{cmyk}{0.62,0.57,0.23,0}
		\definecolor{lightlightgray}{gray}{0.9}
		
		\usepackage{listings} % computer code language formatting
		
		\lstdefinestyle{tex-style} {
		language=TeX,                           % Code langugage
		basicstyle=\ttfamily,                   % Code font, Examples: \footnotesize, \ttfamily
		%keywordstyle=\color{OliveGreen},        % Keywords font ('*' = uppercase)
		commentstyle=\color{gray},              % Comments font
		numbers=none,                           % Line nums position
		numberstyle=\tiny,                      % Line-numbers fonts
		stepnumber=1,                           % Step between two line-numbers
		numbersep=5pt,                          % How far are line-numbers from code
		backgroundcolor=\color{lightlightgray}, % Choose background color
		frame=single,                             % A frame around the code
		tabsize=2,                              % Default tab size
		captionpos=b,                           % Caption-position = bottom
		breaklines=true,                        % Automatic line breaking?
		breakatwhitespace=false,                % Automatic breaks only at whitespace?
		showspaces=false,                       % Dont make spaces visible
		showtabs=false,                         % Dont make tabls visible
		columns=flexible,                       % Column format
		morekeywords={__global__, __device__}  % CUDA specific keywords
		}
		\lstset{style=tex-style}
		
		\usepackage[hyphens]{url}
		\usepackage{hyperref}
		\hypersetup{
		colorlinks=true,
		citecolor=black,
		filecolor=black,
		linkcolor=blue,
		urlcolor=blue
		}
	\end{lstlisting}
	
	\subsection*{Document}
	\begin{latex}	
			\section*{<example-title>}
			
			\subsection*{Description}
			<a-few-words-on-the-example>
			
			\subsection*{Sources}
			\textit{<add reference sources here>}
			
			\subsection*{Used Packages}
			\verb|<state-the-required-packages-here|
			
			\subsection*{Preamble}
			\begin{lstlisting}
			<paste-the-example-related-preamble-code-here>
			\end{lstlisting}
			
			\subsection*{Document}
			\begin{lstlisting}
			<paste-the-document-code-here-for-presentation>
			\end{lstlisting}
			
			\subsection*{Example Result}
			<paste-the-example-document-code-here-for-compilation>
	\end{latex}

\end{document}