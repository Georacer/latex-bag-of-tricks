\documentclass{article}

\usepackage[a4paper, total={7in, 9in}]{geometry}
\pagestyle{empty} % Prevent relative page numbers

%%%%%%%%%%%%%%%%%%%%%%%%%%%%%%%%%%%%%%
%% Default definitions 
%% Define listings format
\usepackage{xcolor} % Required for listings color definitions
\definecolor{Brown}{cmyk}{0,0.81,1,0.60}
\definecolor{OliveGreen}{cmyk}{0.64,0,0.95,0.40}
\definecolor{CadetBlue}{cmyk}{0.62,0.57,0.23,0}
\definecolor{lightlightgray}{gray}{0.9}

\usepackage{listings} % computer code language formatting

\lstdefinestyle{tex-style} {
	language=[LaTeX]TeX,                    % Code langugage
	basicstyle=\ttfamily,                   % Code font, Examples: \footnotesize, \ttfamily
	%keywordstyle=\color{OliveGreen},        % Keywords font ('*' = uppercase)
	commentstyle=\color{gray},              % Comments font
	numbers=none,                           % Line nums position
	numberstyle=\tiny,                      % Line-numbers fonts
	stepnumber=1,                           % Step between two line-numbers
	numbersep=5pt,                          % How far are line-numbers from code
	backgroundcolor=\color{lightlightgray}, % Choose background color
	frame=single,                             % A frame around the code
	tabsize=2,                              % Default tab size
	captionpos=b,                           % Caption-position = bottom
	breaklines=true,                        % Automatic line breaking?
	breakatwhitespace=false,                % Automatic breaks only at whitespace?
	showspaces=false,                       % Dont make spaces visible
	showtabs=false,                         % Dont make tabls visible
	columns=flexible,                       % Column format
	morekeywords={__global__, __device__}  % CUDA specific keywords
}
\lstnewenvironment{latex}
{\lstset{language=[LaTeX]TeX}}
{}
\lstset{style=tex-style}

%% Define URL format
\usepackage[hyphens]{url}
\usepackage{hyperref}
\hypersetup{
	colorlinks=true,
	citecolor=black,
	filecolor=black,
	linkcolor=blue,
	urlcolor=blue
}

\setlength\parindent{0pt}


%%%%%%%%%%%%%%%%%%%%%%%%%%%%%%%%%%%%%%
%% Example-specific packages
%\usepackage[tdf-file]{ticket}
\usepackage[cutmark, rowmode]{ticket}
\usepackage{graphicx}
\usepackage{tikz}


%%%%%%%%%%%%%%%%%%%%%%%%%%%%%%%%%%%%%%
%% Example-specific preamble
% Define key quantities
\newcommand{\labelWidth}{80}
\newcommand{\labelHeight}{60}
\newcommand{\labelSepHeight}{12}
\newcommand{\imageHeight}{45}

% Setup the ticket package lengths
\unitlength = 1mm
\ticketSize{80}{60}
\ticketNumbers{2}{4}
\ticketDistance{0}{0}

% Define the default ticket background
\renewcommand{\ticketdefault}{%
	\put(0,0){%
		\begin{tikzpicture}[x=1mm, y=1mm]
%			% Add temporary grid for easier positioning of elements
%			\draw [gray!50, step=10] (0,0) grid + (\labelWidth,\labelHeight);
%			\node (zeroMark) at (2,2) {0};
			
			% Draw a frame
			\draw (0,0) -- %
			(\labelWidth, 0) --%
			(\labelWidth, \labelHeight) --%
			(0, \labelHeight) --%
			cycle;
			% Draw the horizontal line
			\draw (0, \labelHeight-\labelSepHeight) -- (\labelWidth, \labelHeight-\labelSepHeight);
		\end{tikzpicture}
		}
}

% Define the ticket content
\newcommand{\myLabel}[2]{\ticket{%
		\put(0, 0){%
			\begin{tikzpicture}[x=1mm, y=1mm]
			\node[anchor=center] (title) at (0.5*\labelWidth, \labelHeight-0.6*\labelSepHeight) {\Huge\textbf{#1}};
			\node[anchor=center] (img) at (0.5*\labelWidth, 0.5*\labelHeight-0.5*\labelSepHeight ) {\includegraphics[height=\imageHeight mm]{#2}};
			\node at (80, 60) {};
			\node at (0, 0) {};
			\end{tikzpicture}}%
	}
}

\begin{document}

\section*{Custom Labels}

\subsection*{Description}
I wanted an easy-ish way to make and print labels for some storage boxes I use. I had tried tables in MS Word, and while dimensions and images are straightforward to build, it requires a lot of mouse clicking, dragging and typing, each time I want to make a new label.

The \textit{ticket} pakage provides some amenities to do that.
The philosophy behind it is that
\begin{enumerate}
	\item you define some key lengths in the preamble, pertaining to the label geometry
	\item you define a default ticket background, placing there any content which is common to all tickets
	\item you define a ticket, which places the passed arguments appropriately for each label
\end{enumerate}
Then, you call the \texttt{ticket} command for each new label you want to make.

In theory, this allows you to parse \texttt{.csv} files and turn their contents into labels (e.g. for conference badges).

Read the annotated code for more insight.

\subsection*{Sources}
\url{http://mirror.utexas.edu/ctan/macros/latex/contrib/ticket/doc/manual.pdf}

\subsection*{Used Packages}
\verb|ticket, graphicx, tikz|

\subsection*{Preamble}
\begin{latex}
	% Define key quantities
	% These are used to manually draw geometry
	\newcommand{\labelWidth}{80}
	\newcommand{\labelHeight}{60}
	\newcommand{\labelSepHeight}{12}
	\newcommand{\imageHeight}{45}
	
	% Setup the ticket package lengths
	% These are used by the ticket package
	\unitlength = 1mm % Used unit
	\ticketSize{80}{60} % label dimensions in \unitlength
	\ticketNumbers{2}{4} % number of labels in each page
	\ticketDistance{0}{0} % separation between each label. I selected 0 to cut both edges at once
	
	% Define the default ticket background
	% I used a single tikzpicture to draw the default layout. In theory you could use multiple separate \put commands, but I found the coordinate system awkward.
	\renewcommand{\ticketdefault}{%
		\put(0,0){%
			\begin{tikzpicture}[x=1mm, y=1mm] % It helps turning tikz units into real mms for easier reference
			%			% Add temporary grid for easier positioning of elements
			%			\draw [gray!50, step=10] (0,0) grid + (\labelWidth,\labelHeight);
			%			\node (zeroMark) at (2,2) {0};
			
			% Draw a frame
			\draw (0,0) -- %
			(\labelWidth, 0) --%
			(\labelWidth, \labelHeight) --%
			(0, \labelHeight) --%
			cycle;
			% Draw the horizontal line
			\draw (0, \labelHeight-\labelSepHeight) -- (\labelWidth, \labelHeight-\labelSepHeight);
			\end{tikzpicture}
		}
	}
	
	% Define the ticket content
	\newcommand{\myLabel}[2]{\ticket{%
			\put(0, 0){%
				\begin{tikzpicture}[x=1mm, y=1mm]
				% Draw the lable title
				\node[anchor=center] (title) at (0.5*\labelWidth, \labelHeight-0.6*\labelSepHeight) {\Huge\textbf{#1}};
				% Insert the image
				\node[anchor=center] (img) at (0.5*\labelWidth, 0.5*\labelHeight-0.5*\labelSepHeight ) {\includegraphics[height=\imageHeight mm]{#2}};
				% The two diagnonal points are inserted to create a tikz image as large as the label. In this way, it is correctly placed within the image.
				\node at (\labelWidth, \labelHeight) {};
				\node at (0, 0) {};
				\end{tikzpicture}}%
		}
	}
\end{latex}
%
\subsection*{Document}
\begin{latex}
	% Just use one \myLabel{title}{imageFile} for each label
	\myLabel{Springs}{images/springs.jpg}
	\myLabel{Magnets}{images/magnets}
	\myLabel{Nails}{images/nails}
	\myLabel{Regulators}{images/regulators}
	\myLabel{Sandpaper}{images/sandpaper}
	\myLabel{Screws}{images/screws2}
\end{latex}
%

\clearpage
\subsection*{Result}

\myLabel{Springs}{images/springs.jpg}
\myLabel{Magnets}{images/magnets}
\myLabel{Nails}{images/nails}
\myLabel{Regulators}{images/regulators}
\myLabel{Sandpaper}{images/sandpaper}
\myLabel{Screws}{images/screws2}

\end{document}