\documentclass{article}

\usepackage[a4paper, total={7in, 9in}]{geometry}
\pagestyle{empty}

\usepackage{xcolor} % Required for listings color definitions
\usepackage{listings} % computer code language formatting
\definecolor{Brown}{cmyk}{0,0.81,1,0.60}
\definecolor{OliveGreen}{cmyk}{0.64,0,0.95,0.40}
\definecolor{CadetBlue}{cmyk}{0.62,0.57,0.23,0}
\definecolor{lightlightgray}{gray}{0.9}

\usepackage[hyphens]{url}
\usepackage{hyperref}
\hypersetup{
	colorlinks=true,
	citecolor=black,
	filecolor=black,
	linkcolor=blue,
	urlcolor=blue
}

\lstdefinestyle{tex-style} {
	language=[LaTeX]TeX,                    % Code langugage
	basicstyle=\ttfamily,                   % Code font, Examples: \footnotesize, \ttfamily
	%keywordstyle=\color{OliveGreen},        % Keywords font ('*' = uppercase)
	commentstyle=\color{gray},              % Comments font
	numbers=none,                           % Line nums position
	numberstyle=\tiny,                      % Line-numbers fonts
	stepnumber=1,                           % Step between two line-numbers
	numbersep=5pt,                          % How far are line-numbers from code
	backgroundcolor=\color{lightlightgray}, % Choose background color
	frame=single,                             % A frame around the code
	tabsize=2,                              % Default tab size
	captionpos=b,                           % Caption-position = bottom
	breaklines=true,                        % Automatic line breaking?
	breakatwhitespace=false,                % Automatic breaks only at whitespace?
	showspaces=false,                       % Dont make spaces visible
	showtabs=false,                         % Dont make tabls visible
	columns=flexible,                       % Column format
	morekeywords={__global__, __device__}  % CUDA specific keywords
}
\lstset{style=tex-style}

\usepackage{tikz}
\usetikzlibrary{trees,decorations.markings}

\begin{document}
	
	\section*{File System Structure}
	
	\subsection*{Description}
	A tree file system graphical structure is generated, drawn as a tikz image. The trees tikz library is used to build the structure.
	
	\subsection*{Sources}
	\sloppy
	\url{http://tex.stackexchange.com/questions/306415/connecting-tree-nodes-with-double-arrows-a-la-chef?newsletter=1&nlcode=544695%7c921f}
		
	\subsection*{Used Packages}
	\verb|tikz|
	
	\subsection*{Preamble}
	\begin{lstlisting}
\documentclass[class=article, crop=false]{standalone}
		
\usepackage{tikz}
\usetikzlibrary{trees,decorations.markings}
	\end{lstlisting}
	
	\subsection*{Example Code}
	\begin{lstlisting}
\tikzstyle{every node}=[draw=black,thick,anchor=west]
\tikzstyle{selected}=[draw=red,fill=red!30]
\tikzstyle{optional}=[dashed,fill=gray!50]

\newcommand{\arrowcolor}{red}
\newcommand{\arrowfillcolor}{white}

\pgfdeclarelayer{front}
\pgfsetlayers{main,front}

\makeatletter
\pgfkeys{%
/tikz/path on layer/.code={
\def\tikz@path@do@at@end{\endpgfonlayer\endgroup\tikz@path@do@at@end}%
\pgfonlayer{#1}\begingroup%
}%
}
\makeatother

\begin{tikzpicture}[%
rightarr/.pic={\path[pic actions] (-0.4,0)--(-1,-0.35)--(-1,.35)--cycle;},
grow via three points={one child at (0.5,-0.7) and
two children at (0.5,-0.7) and (0.5,-1.4)},
edge from parent path={(\tikzparentnode.south) |- (\tikzchildnode.west)},
edge from parent/.style={
decoration={
markings,
mark=at position 1 with{\coordinate (0, 0) pic[\arrowcolor,fill=\arrowfillcolor,scale=0.22]{rightarr};},
},
draw = \arrowcolor,
line width = 3pt,
shorten >= 5.7pt,
shorten <= 2pt,
postaction = {decorate},
postaction = {draw,line width=1.4pt,white,path on layer=front},
}]
\node {texmf}
child { node {doc}}     
child { node {fonts}}
child { node {source}}
child { node [selected] {tex}
child { node {generic}}
child { node [optional] {latex}}
child { node {plain}}
}
child [missing] {}                      
child [missing] {}                      
child [missing] {}                      
child { node {texdoc}};
\end{tikzpicture}
	\end{lstlisting}
	
	\subsection*{Example Result}
		\tikzstyle{every node}=[draw=black,thick,anchor=west]
		\tikzstyle{selected}=[draw=red,fill=red!30]
		\tikzstyle{optional}=[dashed,fill=gray!50]
		
		\newcommand{\arrowcolor}{red}
		\newcommand{\arrowfillcolor}{white}
		
		\pgfdeclarelayer{front}
		\pgfsetlayers{main,front}
		
		\makeatletter
		\pgfkeys{%
			/tikz/path on layer/.code={
				\def\tikz@path@do@at@end{\endpgfonlayer\endgroup\tikz@path@do@at@end}%
				\pgfonlayer{#1}\begingroup%
			}%
		}
		\makeatother
		
		\begin{tikzpicture}[%
		rightarr/.pic={\path[pic actions] (-0.4,0)--(-1,-0.35)--(-1,.35)--cycle;},
		grow via three points={one child at (0.5,-0.7) and
			two children at (0.5,-0.7) and (0.5,-1.4)},
		edge from parent path={(\tikzparentnode.south) |- (\tikzchildnode.west)},
		edge from parent/.style={
			decoration={
				markings,
				mark=at position 1 with{\coordinate (0, 0) pic[\arrowcolor,fill=\arrowfillcolor,scale=0.22]{rightarr};},
			},
			draw = \arrowcolor,
			line width = 3pt,
			shorten >= 5.7pt,
			shorten <= 2pt,
			postaction = {decorate},
			postaction = {draw,line width=1.4pt,white,path on layer=front},
		}]
		\node {texmf}
		child { node {doc}}     
		child { node {fonts}}
		child { node {source}}
		child { node [selected] {tex}
			child { node {generic}}
			child { node [optional] {latex}}
			child { node {plain}}
		}
		child [missing] {}                      
		child [missing] {}                      
		child [missing] {}                      
		child { node {texdoc}};
		\end{tikzpicture}
\end{document}