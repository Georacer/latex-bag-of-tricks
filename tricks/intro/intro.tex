\documentclass{article}

\usepackage[a4paper, total={7in, 9in}]{geometry}
\pagestyle{empty}

\usepackage{xcolor} % Required for listings color definitions
\definecolor{Brown}{cmyk}{0,0.81,1,0.60}
\definecolor{OliveGreen}{cmyk}{0.64,0,0.95,0.40}
\definecolor{CadetBlue}{cmyk}{0.62,0.57,0.23,0}
\definecolor{lightlightgray}{gray}{0.9}

\usepackage{listings} % computer code language formatting

\lstdefinestyle{tex-style} {
	language=[LaTeX]TeX,                    % Code langugage
	basicstyle=\ttfamily,                   % Code font, Examples: \footnotesize, \ttfamily
	%keywordstyle=\color{OliveGreen},        % Keywords font ('*' = uppercase)
	commentstyle=\color{gray},              % Comments font
	numbers=none,                           % Line nums position
	numberstyle=\tiny,                      % Line-numbers fonts
	stepnumber=1,                           % Step between two line-numbers
	numbersep=5pt,                          % How far are line-numbers from code
	backgroundcolor=\color{lightlightgray}, % Choose background color
	frame=single,                             % A frame around the code
	tabsize=2,                              % Default tab size
	captionpos=b,                           % Caption-position = bottom
	breaklines=true,                        % Automatic line breaking?
	breakatwhitespace=false,                % Automatic breaks only at whitespace?
	showspaces=false,                       % Dont make spaces visible
	showtabs=false,                         % Dont make tabls visible
	columns=flexible,                       % Column format
	morekeywords={__global__, __device__}  % CUDA specific keywords
}
\lstset{style=tex-style}

\usepackage[hyphens]{url}
\usepackage{hyperref}
\hypersetup{
	colorlinks=true,
	citecolor=black,
	filecolor=black,
	linkcolor=blue,
	urlcolor=blue
}

\begin{document}
	
	\section*{Introduction}
	
	\subsection*{Description}
	At some point during my \LaTeX writing life, I decided to put together a portfolio of cool tricks and tips, which I had used in the past or wanted to apply i the future. Something like an index or a portfolio.
	
	I wanted that document to consist of multiple small parts, one for each tip, and each part should contain both example code and the complied example. A great deal of searching was done to find a technical solution which would allow for a document which would allow for:
	\begin{enumerate}
		\item indexable, possible multi-page examples
		\item examples to be compiled in isolation, free of package conflicts which would surely emerge in such a document
		\item a modular structure, with each example be a short .tex file which would be included to the collection
	\end{enumerate}
	
	I did \href{http://tex.stackexchange.com/questions/316454/include-multiple-independent-snippets-of-latex-code}{a related question} a while back on TeX-StackExchange, but there was no good answer.
	
	Finally, I managed to get a satisfactory result with the \texttt{pdfpages} package. I put together a template document to be used for each example, compile it and then include the output .pdf to the collection document.
	
	The code pattern of the top-level document can be seen below.
	
	Enjoy!
	
	\textbf{NOTE:} Some of the tricks use the \verb|minted| package to typeset \LaTeX code. This requires the addition of the \texttt{-shell-escape} option when building your \LaTeX document.
	For example, my build command in a Linux environment is:\\
	\texttt{"/usr/local/texlive/2016/bin/x86\_64-linux/pdflatex" -synctex=1 -shell-escape}
	
	\hspace{4ex} \texttt{-interaction=nonstopmode \%.tex}
	
	\textbf{WARNING:} Compiling \LaTeX documents with the \texttt{-shell-escape} option enabled is considered a security risk. Only use it to compile documents you trust.
	
		
	\subsection*{Used Packages}
	\verb|pdfpages|
	
	\subsection*{Code}
	\begin{lstlisting}
\documentclass[onepage]{book}

\usepackage{pdfpages}

\begin{document}

\tableofcontents

\chapter{Preliminaries}
\includepdf[pages=-, fitpaper=false, addtotoc={1,section,1,Template,sec:template}]{tricks/introduction/introduction.pdf}

<other-includes>

\end{document}
	\end{lstlisting}
	
	Check on the next section for the template code of each section.

\end{document}