\documentclass{article}

\usepackage[a4paper, total={7in, 9in}]{geometry}
\pagestyle{empty} % Prevent relative page numbers

%%%%%%%%%%%%%%%%%%%%%%%%%%%%%%%%%%%%%%
%% Default definitions 
%% Define listings format
\usepackage{xcolor} % Required for listings color definitions
\definecolor{Brown}{cmyk}{0,0.81,1,0.60}
\definecolor{OliveGreen}{cmyk}{0.64,0,0.95,0.40}
\definecolor{CadetBlue}{cmyk}{0.62,0.57,0.23,0}
\definecolor{lightlightgray}{gray}{0.9}

\usepackage{listings} % computer code language formatting

\lstdefinestyle{tex-style} {
	language=[LaTeX]TeX,                    % Code langugage
	basicstyle=\ttfamily,                   % Code font, Examples: \footnotesize, \ttfamily
	%keywordstyle=\color{OliveGreen},        % Keywords font ('*' = uppercase)
	commentstyle=\color{gray},              % Comments font
	numbers=none,                           % Line nums position
	numberstyle=\tiny,                      % Line-numbers fonts
	stepnumber=1,                           % Step between two line-numbers
	numbersep=5pt,                          % How far are line-numbers from code
	backgroundcolor=\color{lightlightgray}, % Choose background color
	frame=single,                             % A frame around the code
	tabsize=2,                              % Default tab size
	captionpos=b,                           % Caption-position = bottom
	breaklines=true,                        % Automatic line breaking?
	breakatwhitespace=false,                % Automatic breaks only at whitespace?
	showspaces=false,                       % Dont make spaces visible
	showtabs=false,                         % Dont make tabls visible
	columns=flexible,                       % Column format
	morekeywords={__global__, __device__}  % CUDA specific keywords
}
\lstnewenvironment{latex}
{\lstset{language=[LaTeX]TeX}}
{}
\lstset{style=tex-style}

%% Define URL format
\usepackage[hyphens]{url}
\usepackage{hyperref}
\hypersetup{
	colorlinks=true,
	citecolor=black,
	filecolor=black,
	linkcolor=blue,
	urlcolor=blue
}

\setlength\parindent{0pt}


%%%%%%%%%%%%%%%%%%%%%%%%%%%%%%%%%%%%%%
%% Example-specific packages
\usepackage[edges]{forest}

%%%%%%%%%%%%%%%%%%%%%%%%%%%%%%%%%%%%%%
%% Example-specific preamble
% Define custom colors
\definecolor{folderbg}{RGB}{124,166,198}
\definecolor{folderborder}{RGB}{110,144,169}
% Define the folder and file drawing commands
\newlength\Size
\setlength\Size{4pt}
\tikzset{%
	folder/.pic={%
		\filldraw [draw=folderborder, top color=folderbg!50, bottom color=folderbg] (-1.05*\Size,0.2\Size+5pt) rectangle ++(.75*\Size,-0.2\Size-5pt);
		\filldraw [draw=folderborder, top color=folderbg!50, bottom color=folderbg] (-1.15*\Size,-\Size) rectangle (1.15*\Size,\Size);
	},
	file/.pic={%
	\filldraw [draw=folderborder, top color=folderbg!5, bottom color=folderbg!10] (-\Size,.4*\Size+5pt) coordinate (a) |- (\Size,-1.2*\Size) coordinate (b) -- ++(0,1.6*\Size) coordinate (c) -- ++(-5pt,5pt) coordinate (d) -- cycle (d) |- (c) ;
	}
}
% Edit forest options
\forestset{%
	declare autowrapped toks={pic me}{}, % This is an option to hold the value desired for each node. Eventually, this will take values such as pic {file}. By default, it is empty
	pic dir tree/.style={% We define a style for trees of this kind, pic dir tree
		for tree={%
			folder, % Select the directory-style, provided by the folder/edges library
			font=\ttfamily,
			grow'=0,
		},
		before typesetting nodes={% We'll collect the pic me settings while parsing the tree and add them to the edge path by appending the option to the value of edge label
			for tree={%
				edge label+/.option={pic me},
			},
		},
	},
	pic me set/.code n args=2{% pic me set takes 2 arguments. The first is the name of a Forest style to-be-created. The second is the name of a pic such as folder (in the icon sense) or file.
		\forestset{%
			#1/.style={%
				inner xsep=2\Size,
				pic me={pic {#2}},
			}
		}
	},
	pic me set={directory}{folder}, % We just create 2 such styles for now.
	pic me set={file}{file},
}

\begin{document}

\section*{File System Structure 2}

\subsection*{Description}
Another way to plot a file system structure is to take advantage of the \texttt{forest} package. Note that for this specific example \texttt{forest} v2 or greater is required.

\subsection*{Sources}
\url{https://tex.stackexchange.com/questions/328886/making-a-directory-tree-of-folders-and-files?newsletter=1&nlcode=544695%7c921f}\\
\url{https://www.ctan.org/pkg/forest}

\subsection*{Used Packages}
\verb|forest, tikz|

\subsection*{Preamble}
\begin{latex}
%%%%%%%%%%%%%%%%%%%%%%%%%%%%%%%%%%%%%%
%% Example-specific packages
\usepackage[edges]{forest}

%%%%%%%%%%%%%%%%%%%%%%%%%%%%%%%%%%%%%%
%% Example-specific preamble
% Define custom colors
\definecolor{folderbg}{RGB}{124,166,198}
\definecolor{folderborder}{RGB}{110,144,169}
% Define the folder and file drawing commands
\newlength\Size
\setlength\Size{4pt}
\tikzset{%
	folder/.pic={%
		\filldraw [draw=folderborder, top color=folderbg!50, bottom color=folderbg] (-1.05*\Size,0.2\Size+5pt) rectangle ++(.75*\Size,-0.2\Size-5pt);
		\filldraw [draw=folderborder, top color=folderbg!50, bottom color=folderbg] (-1.15*\Size,-\Size) rectangle (1.15*\Size,\Size);
	},
	file/.pic={%
		\filldraw [draw=folderborder, top color=folderbg!5, bottom color=folderbg!10] (-\Size,.4*\Size+5pt) coordinate (a) |- (\Size,-1.2*\Size) coordinate (b) -- ++(0,1.6*\Size) coordinate (c) -- ++(-5pt,5pt) coordinate (d) -- cycle (d) |- (c) ;
	}
}
% Edit forest options
\forestset{%
	declare autowrapped toks={pic me}{}, % This is an option to hold the value desired for each node. Eventually, this will take values such as pic {file}. By default, it is empty
	pic dir tree/.style={% We define a style for trees of this kind, pic dir tree
		for tree={%
			folder, % Select the directory-style, provided by the folder/edges library
			font=\ttfamily,
			grow'=0,
		},
		before typesetting nodes={% We'll collect the pic me settings while parsing the tree and add them to the edge path by appending the option to the value of edge label
			for tree={%
				edge label+/.option={pic me},
			},
		},
	},
	pic me set/.code n args=2{% pic me set takes 2 arguments. The first is the name of a Forest style to-be-created. The second is the name of a pic such as folder or file.
		\forestset{%
			#1/.style={%
				inner xsep=2\Size,
				pic me={pic {#2}},
			}
		}
	},
	pic me set={directory}{folder}, % We just create 2 such styles for now.
	pic me set={file}{file},
}

\end{latex}

\subsection*{Document}
\begin{latex}
\begin{forest}
	pic dir tree, % Import the tree preamble
	where level=0{}{% folder icons by default; override using file for file icons
		directory, % All level 0 nodes will use the directory format
	},
	[system % Now draw the tree
	[config]
	[lib
	[Access]
	[Plugin]
	[file.txt, file] % Override the default folder format and use the file format, as defined above
	]
	[templates]
	[tests]
	]
\end{forest}
\end{latex}

\subsection*{Result}
\begin{forest}
	pic dir tree, % Import the tree preamble
	where level=0{}{% folder icons by default; override using file for file icons
		directory, % All level 0 nodes will use the directory format
	},
	[system % Now draw the tree
		[config]
		[lib
			[Access]
			[Plugin]
			[file.txt, file] % Override the default folder format and use the file format, as defined above
		]
		[templates]
		[tests]
	]
\end{forest}
\end{document}