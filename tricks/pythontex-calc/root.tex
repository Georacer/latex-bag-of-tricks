\documentclass{article}

\usepackage[a4paper, total={7in, 9in}]{geometry}
\pagestyle{empty} % Prevent relative page numbers

%%%%%%%%%%%%%%%%%%%%%%%%%%%%%%%%%%%%%%
%% Default definitions 
\usepackage{xcolor} % Required for listings color definitions
\definecolor{lightlightgray}{gray}{0.9}

%% Specify input format and language
\usepackage[T1]{fontenc}
\usepackage[utf8]{inputenc}
\usepackage[english]{babel}


%% Include minted for source code highlighting
%% WARNING: -shell-escape latex build option required
% Spacing bug and fix: https://tex.stackexchange.com/questions/325939/unwanted-indentation-with-mintinline-possible-bug
\usepackage{tcolorbox}
\tcbuselibrary{minted, skins, breakable}

\newtcblisting{latex}{
	listing engine = minted,
	colback = lightlightgray, 
	colframe = black!70,
	listing only,
	minted style = default,
	minted language = latex,
	left = 1mm,
	breakable, % Options for breaking listing longer than one page
	enhanced jigsaw,
}

\usepackage{minted} % Automatically loads etoolbox
\setminted[latex]{breaklines=True} % Enable line-breaking
%\setminted[latex]{key=value} % Add global language options

%% Define URL format
\usepackage[hyphens]{url}
\usepackage{hyperref}
\hypersetup{
	colorlinks=true,
	citecolor=black,
	filecolor=black,
	linkcolor=blue,
	urlcolor=blue
}

\setlength\parindent{0pt} % Disable paragraph indent


%%%%%%%%%%%%%%%%%%%%%%%%%%%%%%%%%%%%%%
%% Example-specific packages
\usepackage[gobble=auto]{pythontex}
\usepackage{pgfplots} % Used for plotting within LaTeX
\usepackage{graphicx} % Used to import Python-generated pdfs

%%%%%%%%%%%%%%%%%%%%%%%%%%%%%%%%%%%%%%
%% Example-specific preamble
% Pass LaTeX variables into the Python evnironment
\setpythontexcontext{textwidth=\the\textwidth,%
	textheight=\the\textheight%
}

\begin{document}

\section*{Calculations in \LaTeX\ 2: Python Integration}

\subsection*{Description}

The interfaces of the \verb|calc| and \verb|fp| packages may seem cumbersome to some. Furthermore, their capabilities are definitely limited.

\verb|pythontex| offers Python integration for a \LaTeX document. A Python shell is invoked and is passed any code contained within the \verb|pythontex| environments.
The results can be then accessed from the \LaTeX code.

To use \verb|pythontex|, in this case in a Linux environment:
\begin{enumerate}
	\item Write your \LaTeX document, containing any valid \verb|pythontex| code
	\item Compile the \LaTeX document (e.g. \texttt{pdflatex root.tex}) to generate \verb|pythontex|-specific code files
	\item Run the \verb|pythontex| program on the \LaTeX document (i.e. \texttt{pythontex root.tex}).\\
	This should read the generated code files and executre all Python commands
	\item Compile the \LaTeX document anew to include the \verb|pythontex| output
\end{enumerate}

Read through the verbatim example code and the result for more information on the usage of the package.

\textbf{Note 1:} Currently \verb|pythontex| does not support a lower build directory (e.g. ./build/output.pdf). All files must be at the same level as the root document.

\textbf{Note 2:} I have not managed to setup an automated build chain of commands within TeXstudio. Calling \verb|pythontex| from within it sometimes produces runtime errors, especially when making \verb|matplotlib| figures, whereas calling \verb|pythontex| manually from the console runs fine. I have not found the cause.
I tried calling using an external bash script to run \verb|pythontex| from within TeXstudio but this didn't help either.

\textbf{Note 3}: \verb|pythontex| is not compatible with the listings package. If you need to typeset code listings you can use the \verb|minted| package instead. \verb|minted| calls the \verb|pygments| software, which requires the \texttt{-shell-escape} \LaTeX compilation option. This imposes some security risks, so do not use on documents you do not trust.

\textbf{Note 4}: This document uses the \verb|gnuplot| extension of pgfplots, which requires the \texttt{-shell-escape} build option.

\subsection*{Sources}
\url{https://tex.stackexchange.com/questions/397234/how-to-do-mathematical-programming-in-latex?newsletter=1&nlcode=544695%7c921f}\\
\url{https://www.ctan.org/tex-archive/macros/latex/contrib/pythontex}
\url{http://tug.ctan.org/tex-archive/macros/latex/contrib/minted/minted.pdf}

\subsection*{Used Packages}
\verb|graphicx|

\subsection*{Preamble}
\begin{latex}
%%%%%%%%%%%%%%%%%%%%%%%%%%%%%%%%%%%%%%
%% Example-specific packages
\usepackage[gobble=auto]{pythontex}
\usepackage{pgfplots} % Used for plotting within LaTeX
\usepackage{graphicx} % Used to import Python-generated pdfs

%%%%%%%%%%%%%%%%%%%%%%%%%%%%%%%%%%%%%%
%% Example-specific preamble
% Pass LaTeX variables into the Python evnironment
\setpythontexcontext{textwidth=\the\textwidth,%
	textheight=\the\textheight%
}
\end{latex}

\subsection*{Document}
\begin{latex}
Python\TeX provides a multitude of environments which allow \LaTeX to interact with Python.

The \mintinline{latex}{\pyb}/\mintinline{latex}{pyblock} environment typesets the passed code into the \LaTeX document and also executes it. However, it will not print out any results.

\pyb{my_str='hi'}

\begin{pyblock}
	my_str+=' there'
	print('Python says: {0}!'.format(my_str))
\end{pyblock}

You can access the last printout done by Python using the \mintinline{latex}{\printpythontex} command.

\printpythontex

The \mintinline{latex}{\pyc}/\mintinline{latex}{pycode} environment both executes code and typesets results, but does not typeset the code itself.

\pyc{my_str="calculations"}

\begin{pycode}
	my_str="calculations"
	print("Let's do some {0}".format(my_str))
\end{pycode}

You can use the \mintinline{latex}{\pys}/\mintinline{latex}{pysub} environment to substitute Python variable values into \LaTeX code, using the \verb|!{...}| format.

In this example, \verb|sympy| expression manipulation is used to generate a function expression, which is then plotted using \verb|gnuplot|.

\begin{pycode}
	from sympy import *
	x = symbols('x')
	f = integrate(cos(x)*sin(x), x)
\end{pycode}

\begin{pysub}
	\begin{tikzpicture}
	\begin{axis}[xlabel=$x$,ylabel=$y$,samples=200,no markers,title=$!{latex(f)}$]
	\addplot[black] gnuplot {!{f}};
	\end{axis} 
	\end{tikzpicture}
\end{pysub}

The \texttt{r} string prefix prevents escaping characters. This can be used to effectively perform \LaTeX code generation within  a \mintinline{latex}{pycode} block.

\begin{pycode}
	print(r'\begin{center}')
		print(r'\textit{A message from Python!}')
		print(r'\end{center}')
\end{pycode}

Now, let us focus on actual calculations done with \verb|pythontex|.

\textbf{Example 1: Typical arithmetic}
\begin{pyblock}
	x1 = 5.1
	x2 = 7.4
	y = 3*x1**2 + 5*x2**0.5
\end{pyblock}
The answer is \py{y}.

~\\
\textbf{Example 2: Math environment code generation}

\pys{$1 + 1 = !{1+1}$}

~\\
\textbf{Example 3: String manipulation}

\begin{pyblock}
	username = "Doctor"
	password = "Zee"
\end{pyblock}
\begin{pysub}
	Your username is: \textit{!{username}}\\
	Your password is: \textit{!{password}}
\end{pysub}

~\\
\textbf{Example 4: Using} \verb|matplotlib| \textbf{to generate figures}

The \mintinline{latex}{pylabcode} block automatically imports the \mintinline{python}{matplotlib} module.
First of all, notice the usage of the \texttt{textwidth} variable, which was passed to the Python environment at the preamble, to control the size of the figure.
Following that, the matplotlib commands are juxtaposed.
Finally, a .pdf figure is saved with the \mintinline{python}{savefig} command.

\textbf{Note:} Your initial \LaTeX must have any \mintinline{latex}{\includegraphics} commands commented out, because these files have not yet been created by \verb|pythontex|. Once the image files are created, you can un-comment any image include commands.

\begin{pylabcode}
	figwidth = 0.4*pytex.pt_to_in(pytex.context.textwidth)
	figheight = 3/4*figwidth
	rc('text', usetex=True)
	rc('font',**{'family':'serif', 'serif':['Times']})
	rc('font', size=10.0)
	rc('legend', fontsize=10.0)
	x = linspace(0, 3*pi)
	figure(figsize=(figwidth, figheight))
	plot(x, sin(x), label='$\sin(x)$')
	plot(x, sin(x)**2, label='$\sin^2(x)$',	linestyle='dashed')
	xlabel(r'$x$-axis')
	ylabel(r'$y$-axis')
	xticks(arange(0, 4*pi, pi), ('$0$','$\pi$', '$2\pi$', '$3\pi$'))
	axis([0, 3*pi, -1, 1])
	legend(loc='lower right')
	savefig('myplot.pdf', bbox_inches='tight')
\end{pylabcode}


\begin{center}
	\includegraphics{myplot.pdf}
\end{center}
\end{latex}

\subsection*{Result}

Python\TeX provides a multitude of environments which allow \LaTeX to interact with Python.

The \mintinline{latex}{\pyb}/\mintinline{latex}{pyblock} environment typesets the passed code into the \LaTeX document and also executes it. However, it will not print out any results.

\pyb{my_str='hi'}

\begin{pyblock}
	my_str+=' there'
	print('Python says: {0}!'.format(my_str))
\end{pyblock}

You can access the last printout done by Python using the \mintinline{latex}{\printpythontex} command.

\printpythontex

The \mintinline{latex}{\pyc}/\mintinline{latex}{pycode} environment both executes code and typesets results, but does not typeset the code itself.

\pyc{my_str="calculations"}

\begin{pycode}
	my_str="calculations"
	print("Let's do some {0}".format(my_str))
\end{pycode}

You can use the \mintinline{latex}{\pys}/\mintinline{latex}{pysub} environment to substitute Python variable values into \LaTeX code, using the \verb|!{...}| format.

In this example, \verb|sympy| expression manipulation is used to generate a function expression, which is then plotted using \verb|gnuplot|.

\begin{pycode}
	from sympy import *
	x = symbols('x')
	f = integrate(cos(x)*sin(x), x)
\end{pycode}

\begin{pysub}
	\begin{tikzpicture}
	\begin{axis}[xlabel=$x$,ylabel=$y$,samples=200,no markers,title=$!{latex(f)}$]
	\addplot[black] gnuplot {!{f}};
	\end{axis} 
	\end{tikzpicture}
\end{pysub}

The \texttt{r} string prefix prevents escaping characters. This can be used to effectively perform \LaTeX code generation within  a \mintinline{latex}{pycode} block.

\begin{pycode}
	print(r'\begin{center}')
	print(r'\textit{A message from Python!}')
	print(r'\end{center}')
\end{pycode}

Now, let us focus on actual calculations done with \verb|pythontex|.

\textbf{Example 1: Typical arithmetic}
\begin{pyblock}
	x1 = 5.1
	x2 = 7.4
	y = 3*x1**2 + 5*x2**0.5
\end{pyblock}
The answer is \py{y}.

~\\
\textbf{Example 2: Math environment code generation}

\pys{$1 + 1 = !{1+1}$}

~\\
\textbf{Example 3: String manipulation}

\begin{pyblock}
	username = "Doctor"
	password = "Zee"
\end{pyblock}
\begin{pysub}
	Your username is: \textit{!{username}}\\
	Your password is: \textit{!{password}}
\end{pysub}

~\\
\textbf{Example 4: Using} \verb|matplotlib| \textbf{to generate figures}

The \mintinline{latex}{pylabcode} block automatically imports the \mintinline{python}{matplotlib} module.
First of all, notice the usage of the \texttt{textwidth} variable, which was passed to the Python environment at the preamble, to control the size of the figure.
Following that, the matplotlib commands are juxtaposed.
Finally, a .pdf figure is saved with the \mintinline{python}{savefig} command.

\textbf{Note:} Your initial \LaTeX must have any \mintinline{latex}{\includegraphics} commands commented out, because these files have not yet been created by \verb|pythontex|. Once the image files are created, you can un-comment any image include commands.

\begin{pylabcode}
	figwidth = 0.4*pytex.pt_to_in(pytex.context.textwidth)
	figheight = 3/4*figwidth
	rc('text', usetex=True)
	rc('font',**{'family':'serif', 'serif':['Times']})
	rc('font', size=10.0)
	rc('legend', fontsize=10.0)
	x = linspace(0, 3*pi)
	figure(figsize=(figwidth, figheight))
	plot(x, sin(x), label='$\sin(x)$')
	plot(x, sin(x)**2, label='$\sin^2(x)$',	linestyle='dashed')
	xlabel(r'$x$-axis')
	ylabel(r'$y$-axis')
	xticks(arange(0, 4*pi, pi), ('$0$','$\pi$', '$2\pi$', '$3\pi$'))
	axis([0, 3*pi, -1, 1])
	legend(loc='lower right')
	savefig('myplot.pdf', bbox_inches='tight')
\end{pylabcode}


\begin{center}
	\includegraphics{myplot.pdf}
\end{center}


\end{document}